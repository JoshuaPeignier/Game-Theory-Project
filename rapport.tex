% !TEX spellcheck = en-US
\documentclass{article}
\usepackage[T1]{fontenc}
\usepackage{url,color,amstext,amsmath,amssymb,amsthm,tikz,tikz-inet,pgf,pgfplots}
\usepackage{fp,float}

\def\colorpat{red}
\def\colorbruno{blue} % Bruno plus doux que Patrick

\newif\ifnotes\notestrue
%\notesfalse             %%  Uncomment this line to hide footnotes.  <----
%
\def\boxnote#1#2{\ifnotes\fbox{\footnote{\ }}\ \footnotetext{ From #1: #2}\fi}
%
\newcommand\mpat[1]{{\color{\colorpat} #1}}
\newcommand\spat[1]{{\scriptsize{\color{violet} #1}}}
\newcommand\hspat[1]{}
\def\pat#1{\boxnote{Patrick}{\color{\colorpat}#1}}
\def\hpat#1{}
\newcommand\mbruno[1]{{\color{\colorbruno} #1}}
\def\bruno#1{\boxnote{Bruno}{\color{\colorbruno}#1}}
\def\hbruno#1{}

\newtheorem{proposition}{Proposition}
\newtheorem{lemma}{Lemma}
\newtheorem{definition}{Definition}
\newcommand\ind[1]{1\hspace{-.55ex}\mbox{\textnormal l}_{\{#1\}}}  %fonction indicatrice

\gdef\figwidth{7cm}
\gdef\figheight{6cm}
\gdef\bc{b^{\rm c}}
\gdef\bv{b^{\rm v}}
\gdef\subc{{\tilde b}^{\rm c}}
\gdef\subv{{\tilde b}^{\rm v}}
\gdef\suv{{\tilde v}}
\gdef\pc{p^{\rm c}}
\gdef\pv{p^{\rm v}}
\gdef\CTR{\text{\footnotesize CTR}}
\gdef\dd{\mathrm{d}}
\gdef\define{:=}
%\gdef\Ealph{{\mathbb E}[\alpha]}
%\gdef\Ealphbet{{\mathbb E}[\alpha\beta]}
\gdef\Ealph{\bar{\alpha}}
\gdef\Ebet{\bar{\beta}}
\gdef\Ealphbet{\overline{\alpha\beta}}

\renewcommand\S[1]{S_{\mathrm{#1}}}
\newcommand\Rev[1]{R_{\mathrm{#1}}}

\def\E {{\mathbb E}}
\def\EE {{\mathbb E}}
\def\FF {{\mathbb F}}
\def\II {{\mathbb I}}
\def\LL {{\mathbb L}}
\def\NN {{\mathbb N}}
\def\PP {{\mathbb P}}
\def\RR {{\mathbb R}}
\def\ZZ {{\mathbb Z}}

\pgfcreateplotcyclelist{\mylist}{
	{blue,thick},
	{red,thick,dash pattern=on 6pt off 1pt on 2pt off 1pt}, 
	{blue,thick,densely dashed},
	{red,thick,dashed},
	{blue,thick,densely dotted},
	{red,thick,dotted}, 
}


\title{Community Network in Competition with ISPs}

\author{Patrick Maill\'e\\ Institut Mines-T\'el\'ecom / T\'el\'ecom Bretagne \\ 2, rue de la Châtaigneraie, 35576 Cesson S\'evign\'e Cedex, France, \\
       \tt{patrick.maille@telecom-bretagne.eu}
\and Bruno Tuffin \\ Inria Rennes Bretagne Atlantique \\ Campus Universitaire de Beaulieu, 35042 Rennes Cedex, France \\
       \tt{bruno.tuffin@inria.fr}
       }
\begin{document}
\maketitle

\begin{abstract}
XXXXXXX
\end{abstract}

\section{Estelle's prior notes}
This is a temporary section where I recall any questionnable decisions I made so that we will remember to debate them.

We will suppose that $\int_\Omega f(v) g(v|u) dv = q_{max} < +\infty$.

\section{Introduction}


XXXXX Intro community networks

XXXXXX Models: cf Marbach and Hubaux

XXXX Our contribution instead: investigate the impact mobility and user density on community networks, with the competitive presence of an ISP.


\section{Model}

Consider a continuum of users characterized by their type $u$. This type typically defines a home location.
There is a density $f(u)$ in terms of parameter $u$ running over space $\Omega$.


We consider two ways for users to access the Internet: through a regular ISP, assumed to fully cover space, or through a community network. Let $D\subset \Omega$ be the set (domain) of users subscribing to  the community network.
 
Each user $u$ has an average number of communication requests $m(u)$. Define $g(v|u)$ as the density that a given request from $u$is happening when at $v$.
It models the mobility behavior of user $u$.
Then the number (density) of requests at $u$ from the community network is
$$
n(u) =\int_D g(u|v) m(v) f(v) dv.
$$

A user $u$ chooses to subscribe to the ISP or the community network depending on the subscription price and a notion of quality which we define as the proportion of (or probability that) requests will be fulfilled. If made through the ISP, this probability is $1$. If made through the community network, it is 
$$
q_u=\int_D g(v|u)  f(v) dv.
$$  

Let $a$ be a parameter representing the sensitivity to quality. and $p_{I}$ (respectively $p_C$) the subscription price to the ISP (respectively the community network).
 
 The utilities of $u$ for using the community network, the ISP, or none of them  are respectively
 \begin{eqnarray*}
 U_C(u) & = & a q_u -p_C  - c n(u)\\
 U_I(u) & = & a -p_I \\
 U_\emptyset (u)  & = & 0
 \end{eqnarray*}
 where $c n(u)$ models the disturbance from requests of others $u$ has to serve, $c$ corresponding to a unit cost.
 
 User $u$'s choice is the one with the maximal utility.
 
\subsubsection{Estelle 's notes about the model} 
\subsection{Sub-model}
For obvious reasons $p_i > 0$ and $p_c > 0$ (charity is not a valid option).

As long as $U_I > 0$ a type u user is sure to choose $I$ it if he did not choose $C$.
This means all users suscribe to one of the two services.

Therefore $C$ chosen iff $U_c > U_i$.

We define $p$ the diffrence of prices $p:= p_i - p_c$
Therefore $p < a$.

We normalize f so that $\int_\Omega f(v) dv =1$.

We will use the notations
\begin{itemize}
\item $d_c := \int_D f(v)dv$
\item $d_i := 1 - d_c$
\item $q_{u,max} := \int_\Omega g(v|u) f(v)dv$
\end{itemize} 
 
 
\section{Game definition}
 
 
We thus end up with a multilevel game where:
\begin{enumerate}
\item The ISP and community network play on the subscription prices, in order to maximize revenues (expressed as the product of price and mass of users). 
\item Given the allocated qualities, users choose their network. 
\end{enumerate}
The game is played by backward induction, meaning that even if ISPs play first, they will make their decision strategically, by anticipating what will be the subsequent decision of users.


 
\section{First case study}
 
 We first consider the simplest situation where the mobility pattern is the same for all users:
 $$ g(v|u)=g(v).$$

In that case, $q_u$ does not depend on $u$:
$$
q_u=\int_D g(v) f(v) dv $$
and $n(u)=\in_D g(u) m(v) f(v) dv = g(u) \int_D m(v) f(v) dv.$

\subsection{A problimatic example}
First, we must detemine if the values of $p_i$ and $p_c$ are sufficiant to determine the players revenues. We wish to show in this section that this isn't the case.

To prove the previous statement we will choose a simple model with fixed values for $p_i$ and $p_c$ and show that this can result in sevral diffrent revenues.


In our model, $p = a -\epsilon$. ($p_i$ and $p_c$ were preditermined and do not vary in this section). All users be considered identical : $g=cst, f=cst, m=cst ...$,
 please note that in this situation all users will make the same choice (they are identical). Therefore $U_c = a g d_c -p_c -c m g d_c = g d_c (a-cm) -p_c.$
If $cm > a$ then $d_c = 0$ since no one will choose a negative utility.

Else case 1, $d_c = 1$.
Then $U_c -U_i = g (a - cm) - a + a - \epsilon = g (a -cm) - \epsilon> 0$. This is coherent, if everyone chose $C$ then that means there are many networks to choose from so $C$ is the best choice.

Else case 2, $d_c = 0$.
Then $U_c - U_i = -\epsilon < 0$. This is also coherent, when no one is part of the community the community network is not intresting.


This example shows that there can be diffent densities $d_c$ for a single price situation $p$. The set $Q(p)= Q(p_i,p_c)$ is the set of these densities. We will therefore nead to express our problem according to an other variable then $p$.

\subsection{Examining the $g(u|v) = g(u)$ case}
We define $b = c \int_D m(v) f(v) dv$
A expressed previously, $u$ chooses $C$ if

$$U_c > U_i \Leftrightarrow  a (q_u -1) +p > c n(u) = b g(u)$$

Therefore we will define the fallowing
\begin{itemize}
\item $D_u(k) := \{u | g(u) < k\}$
\item $q_u(k) := \int_{D_u(k)} f(v) g(v) dv$
\item $b_u(k) :=\int_{D_u(k)} f(v) m(v) dv$.
\item $d_u(k) = \int_{D_u(k)} f(v) dv$ 
\item $\Phi(k,p) = a (q_u(k) -1) + p - b(k) k$
\end{itemize}

Therefore we can say that $D_u(k) = D <=> \Phi(k,p) = 0$.
 
This is intresting since it means we now know that $p$ is a function of $k$.
 $$p (k) = b(u) k - a(q_u(k) -1)$$

We can also use the continuity of $\Phi$.

Recall we assumed that $q_max$ was finite in other words that there is a limit to how good a service the community service provides.

\begin{itemize}
\item $\Phi$ is contenuous
\item $\Phi(0,p) = p - a < 0$.
\item $\Phi(+\infty,p) = -\infty$
\end{itemize}

This shows that the number of domains for a given price $p$ is finite. From this resuts that there are a finite number of contenuous functions $f$ such as $f(p(k)) = k$.

\subsubsection{From $p$ to the rewards}
Too options one we have $I$ that chooses to take $p_i = a$ then by evaluations of 

We have $p_i = p + p_c$ et $0 < p_i < a$.

$$p(k) = p_i + p_c$$

$$G_i (p_i,k) = \{ p_i * (1 - d_u(k)) \}$$

$$G_c (p_c,k) = \{ p_c* d_u(k)) \}$$

We will use our $f$ functions.

$$G_i (p_i,p_c) = \{ F_j : p_i \rightarrow p_i *(1 - d_u( f(p))) | \forall k, \exists p(k) \Rightarrow f(p(k)) = k \}$$

$$H_i(p_i,p_c) = \dfrac{d G_i(p_i,p_c)}{dp_i} = \{F'_j : p_i \rightarrow (1 - d_u(f(p))) - p_i*\dfrac{dd_u(f(p))}{dp} \}$$

Note that for all $j$ we have $F_j'$ that originates from $p$ and $F_j$.

$$G_c (p_i,p_c) = \{ F_j : p_c \rightarrow p_c *d_u(f(p)) | \forall k, \exists p(k) \Rightarrow f(p(k)) = k \}$$

$$H_c(p_i,p_c) =\dfrac{d G_c(p_i,p_c)}{dp_i} = \{F'_j : p_c \rightarrow d_u(f(p)) - p_c*\dfrac{dd_u(f(p))}{dp} \}$$

So as to be able to exploit the values we compare $G_i$ and $G_c$ using the same $f$ 

Therefore we look for every $f$ the conditions on $p_i$ and $p_c$ that result in $H_i = 0$ and $H_c = 0$. Once this is done we can deduce what the co-local-maximum of $G_i$ and $G_c$ are. Therefore we can find all the Nash equilibrum for the given $f$.

By analysing all $f$'s and the results of a given $(p_i,p_c)$ that is a Nash equilibrum for one $f$ in all other $f$ functions we know all the parametters of the problem.




XXXX TBC

 \section{Second case study}
 
 We consider a domain made of two cities separated by the country side, and the same behavior of all users in an area.
 Formally, $\Omega=\Omega_{c_1}\cup \Omega_{c_2}\cup \Omega_s$ where $\Omega_{c_k}$ is for the population of City~$k$ and $\Omega_s$ for the population of the countryside.
 

 For $k\in \{c_1,c_2,s\}$ and $u\in \Omega_k$, let $g_k(v|u)$ be the mobility behavior of $u$.
 
 
 XXXXXX TBC
 
 
\bibliographystyle{plain}
\bibliography{biblio}

\end{document}

